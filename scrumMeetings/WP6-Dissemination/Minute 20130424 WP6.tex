
\documentclass[a4paper, 11pt]{article}
\usepackage[ascii]{inputenc}
\usepackage{supertabular}
\usepackage[ngerman]{babel}
\usepackage{amsmath}
\usepackage{amssymb,amsfonts,textcomp}
\usepackage {geometry}
\geometry{a4paper,top=25mm,left=30mm,right=25mm,bottom=30mm}
\usepackage{color}
\usepackage{array}
\usepackage{hhline}
\usepackage{hyperref}
\hypersetup{colorlinks=true, linkcolor=blue, citecolor=blue, filecolor=blue, urlcolor=blue}


\begin{document}
{\begin{center}\huge\bf openETCS WP7 Meeting Minutes\end{center}}
\section{Meeting Information}

\renewcommand{\arraystretch}{1.5}
\begin{supertabular}{m{.2\textwidth}m{.8\textwidth}}
%\hline
Subject & WP6 Weekly Scrum\\
Date \& time & 2013-05-24, 10:00h--10:15h\\
Location & Telco and Goto-Meeting\\
Called up by & Stefan Rieger\\
Participants & WP6,
Ayse Yurdakul,
Cecile Braunstein,
Cyril Cornu,
Frank Golatowski,
Pierre Frauncois Jauquet,
Marc Behrens,
Stefan Rieger,
Uwe Steinke,
Fausto Cochetti,
Begona Laibarra,
KLaus-R\"udiger Hase,
Peter Mahlmann,
Baseliyos Jacob
\\


Minutes by & Peter Mahlmann\\

%\hline
\end{supertabular}
\renewcommand{\arraystretch}{1.0}

%\line(1,0){440}

\section{Agenda}
\begin{itemize}
\item Management of openETCS publications and glossary
\end{itemize}

\section{Discussion}

\begin{itemize}
\item Zotero (available as stand-alone application and browser plugin) will be considered as a tool for the management of publications. A Wiki page describing the use of Zotero will be set up by Stefan Rieger.
\item It was proposed to put all openETCS publications on github. Maybe publications who must not be made available to the public due to copyright reasons will be put on sharepoint or a private repository.

\end{itemize}

%\line(1,0){440}
\section{Notes}

\end{document}
