% This is an automatic generated Latex document based on the Iglos Glossary by Technische Universit�t Braunschweig. 
 %This Version has been released at the 11-Sep-2013 
 % new item : 
 % \newglossaryentry{?label?}{?key-val list?}
 %--------------------
 % usage in latex file
 %--------------------
%     Preamble
%--------------------
% \package[section, % add the glossary to the table of content
%            description,% acronyms have a user-supplied description,
 % style=superheaderborder, % table style
 % nonumberlist % no page number
 % ]{glossaries}
%\renewcommand*{\glossaryname}{List of Terms}
 %\makeglossaries
 %\loadglsentries{wp7_glossary}
 %--------------------
 %     reference
 %--------------------
 % \gls{label}
 %--------------------
 % compile command
 %--------------------
 % makeglossaries
 %--------------------);
 

 %Start of Glossary Terms
\newglossaryentry{plan of dissemination and exploitation}{
 name={plan of dissemination and exploitation},
 description={}
 }
 
\newglossaryentry{functional specification}{
 name={functional specification},
 description={description of the system function}
 }
 
\newglossaryentry{Radio Block Centre}{
 name={Radio Block Centre},
 description={A centralised safety unit working with an interlocking(s) to establish and control train separation. Receives location information via radio from trains and sends movement authorities via radio to trains.}
 }
 
\newacronym{RBC} % label 
{RBC} %abbreviation 
 {Radio Block Centre} %long form 
 
\newglossaryentry{project management}{
 name={project management},
 description={planning, organizing, monitoring, controlling and reporting of all aspects of a project and the motivation of all those involved in it to achieve the project objectives}
 }
 
\newacronym{PM} % label 
{PM} %abbreviation 
 {project management} %long form 
 
\newglossaryentry{man-machine interface}{
 name={man-machine interface},
 description={The ERTMS / ETCS train borne device to enable communication between ERTMS / ETCS and the train driver.}
 }
 
\newacronym{MMI} % label 
{MMI} %abbreviation 
 {man-machine interface} %long form 
 
\newglossaryentry{high-speed line}{
 name={high-speed line},
 description={A section of route forming part of the European High Speed Rail Network and any additional routes specified as such by national administrations.}
 }
 
\newacronym{HSL} % label 
{HSL} %abbreviation 
 {high-speed line} %long form 
 
\newglossaryentry{subsystem}{
 name={subsystem},
 description={Combination of equipment, units and assemblies etc. that performs an operational function and that is a major subdivision of the system.}
 }
 
\newacronym{SS_2} % label 
{SS} %abbreviation 
 {subsystem} %long form 
 
\newglossaryentry{unit}{
 name={unit},
 description={a collection of associated control modules and/or equipment modules and other process equipment in which one or more major processing activities can be conducted.}
 }
 
\newglossaryentry{system}{
 name={system},
 description={set of interrelated or interacting elements}
 }
 
\newglossaryentry{railway undertaking}{
 name={railway undertaking},
 description={Any public or private undertaking licensed according to applicable Community legislation, the principal business of which is to provide services for the transport of goods and/or passengers by rail with a requirement that the undertaking must ensure traction; this also includes undertakings which provide traction only.}
 }
 
\newacronym{RU} % label 
{RU} %abbreviation 
 {railway undertaking} %long form 
 
\newglossaryentry{function}{
 name={function},
 description={the action or actions that a product is designed to perform.}
 }
 
\newglossaryentry{train protection system}{
 name={train protection system},
 description={}
 }
 
\newacronym{TPS} % label 
{TPS} %abbreviation 
 {train protection system} %long form 
 
\newglossaryentry{equipment}{
 name={equipment},
 description={The rolling stock of a carrier.}
 }
 
\newglossaryentry{product}{
 name={product},
 description={result of a process}
 }
 
\newglossaryentry{control device}{
 name={control device},
 description={a control device is connected to the interface and sends commands in order to control other devices (for example lamp control gear) connected to the same interface.}
 }
 
\newacronym{CD_2} % label 
{CD} %abbreviation 
 {control device} %long form 
 
\newglossaryentry{rolling stock}{
 name={rolling stock},
 description={A railroad vehicle that is not a locomotive; 'railroad car'. (US) Any railroad car and/or locomotive.}
 }
 
\newacronym{RST} % label 
{RST} %abbreviation 
 {rolling stock} %long form 
 
\newglossaryentry{approval}{
 name={approval},
 description={Permission for a product or process to be marketed or used for stated purposes or under stated conditions.}
 }
 
\newglossaryentry{test automatisation}{
 name={test automatisation},
 description={}
 }
 
\newacronym{TA} % label 
{TA} %abbreviation 
 {test automatisation} %long form 
 
\newglossaryentry{railway infrastructure operator}{
 name={railway infrastructure operator},
 description={Any public or private undertaking licensed in accordance with the European Communities (Licensing of Railway Undertakings) Regulations 2003 (S.I. No. 537 of 2003) or otherwise by law, the principal activity of which is to provide services for the transport of goods or passengers or both by rail, and any other public or private undertaking, the activity of which is to provide transport of goods or passengers or both by rail, on the basis that the undertaking must ensure traction including an undertaking which provides traction only.}
 }
 
\newacronym{RIO} % label 
{RIO} %abbreviation 
 {railway infrastructure operator} %long form 
 
\newglossaryentry{railway system}{
 name={railway system},
 description={The totality of the subsystems for structural and operational areas, as defined in Directives 96/48/EC and 2001/16/EC, as well as the management and operation of the system as a whole.}
 }
 
\newacronym{RS} % label 
{RS} %abbreviation 
 {railway system} %long form 
 
\newglossaryentry{design}{
 name={design},
 description={the activity applied in order to analyse and transform specified requirements into acceptable design solutions which have the required safety integrity}
 }
 
\newglossaryentry{open-source license}{
 name={open-source license},
 description={}
 }
 
\newglossaryentry{open-source tool}{
 name={open-source tool},
 description={}
 }
 
\newacronym{OSL, open-source tool} % label 
{OSL, open-source tool} %abbreviation 
 {open-source tool} %long form 
 
\newglossaryentry{multiple unit}{
 name={multiple unit},
 description={a self-propelled rail vehicle that can be joined with compatible others and controlled from a single driving station. The sub-classes of this type of vehicle}
 }
 
\newacronym{MU} % label 
{MU} %abbreviation 
 {multiple unit} %long form 
 
\newglossaryentry{remote access}{
 name={remote access},
 description={communication from outside the receiving unit, through a data link that gives the possibility to control/influence the deposit sequence of a distributed system}
 }
 
\newacronym{RA} % label 
{RA} %abbreviation 
 {remote access} %long form 
 
\newglossaryentry{tool}{
 name={tool},
 description={Software utility}
 }
 
\newglossaryentry{progress}{
 name={progress},
 description={}
 }
 
\newglossaryentry{vital}{
 name={vital},
 description={A description applied to equipment whose correct operation is essential to the integrity of the signalling system. Most vital equipment is designed to fail-safe principles - a wrong side failure of vital equipment could directly endanger rail traffic.}
 }
 
\newglossaryentry{system safety integrity level}{
 name={system safety integrity level},
 description={number which indicates the required degree of confidence that a system will meet its specified safety features}
 }
 
\newglossaryentry{platform}{
 name={platform},
 description={a set of sub-systems and interfaces that form a common structure from which a set derivative products can be efficiently developed and produced.}
 }
 
\newglossaryentry{implementer}{
 name={implementer},
 description={one or more persons assigned by the Design Authority to transform specified designs into their physical realisation}
 }
 
\newacronym{IMP} % label 
{IMP} %abbreviation 
 {implementer} %long form 
 
\newglossaryentry{software life-cycle}{
 name={software life-cycle},
 description={activities occurring during a period of time that starts when software is conceived and ends when the software is no longer available for use. The software lifecycle typically includes a requirements phase, development phase, test phase, integration phase, installation phase and a maintenance phase}
 }
 
\newacronym{SLC} % label 
{SLC} %abbreviation 
 {software life-cycle} %long form 
 
\newglossaryentry{software safety integrity level}{
 name={software safety integrity level},
 description={classification number which determines the techniques and measures that have to be applied in order to reduce residual software faults to an appropriate level}
 }
 
\newglossaryentry{software maintenance}{
 name={software maintenance},
 description={Action, or set of actions, carried out on software after its acceptance by the final user. The aim is to improve, increase and/or correct its functionality}
 }
 
\newacronym{SM} % label 
{SM} %abbreviation 
 {software maintenance} %long form 
 
\newglossaryentry{validation}{
 name={validation},
 description={activity of demonstration, by analysis and test, that the product meets, in all respects, its specified requirements}
 }
 
\newglossaryentry{software maintainability}{
 name={software maintainability},
 description={capability of a system to be modified to correct faults, improve performance or other attributes, or adapt it to a different environment}
 }
 
\newglossaryentry{software}{
 name={software},
 description={intellectual creation comprising the programs, procedures, rules and any associated documentation pertaining to the operation of a system}
 }
 
\newacronym{SW} % label 
{SW} %abbreviation 
 {software} %long form 
 
\newglossaryentry{safety}{
 name={safety},
 description={freedom from unacceptable levels of risk}
 }
 
\newglossaryentry{interoperability}{
 name={interoperability},
 description={means the ability of the trans-European high-speed rail system to allow the safe and uninterrupted movement of highspeed trains that accomplish the specified levels of performance}
 }
 
\newglossaryentry{requirements traceability}{
 name={requirements traceability},
 description={objective of requirements traceability is to ensure that all requirements can be shown to have been properly met}
 }
 
\newacronym{RT} % label 
{RT} %abbreviation 
 {requirements traceability} %long form 
 
\newglossaryentry{safety-related software}{
 name={safety-related software},
 description={software which carries responsibility for safety}
 }
 
\newglossaryentry{risk}{
 name={risk},
 description={combination of the frequency, or probability, and the consequence of a specified hazardous event}
 }
 
\newglossaryentry{avionic}{
 name={avionic},
 description={(derived from the expression ?aviation electronics?), the development and production of electronic instruments for use in aviation and astronautics. The term also refers to the instruments themselves. Such instruments consist of a wide variety of control, performance, communications, and radio navigation devices and systems.}
 }
 
\newglossaryentry{safety critical}{
 name={safety critical},
 description={term applied to any condition, event, operation, process, or item whose proper recognition, control, performance, or tolerance is essential to safe system operation and support.}
 }
 
\newglossaryentry{usability}{
 name={usability},
 description={extent to which a system, product or service can be used by specified users to achieve specified goals with effectiveness, efficiency and satisfaction in a specified context of use}
 }
 
\newglossaryentry{end-user}{
 name={end-user},
 description={consumer (private person, enterprise, utility etc.) using fuel for energy purposes}
 }
 
\newglossaryentry{homologation}{
 name={homologation},
 description={}
 }
 
\newglossaryentry{embedded control system}{
 name={embedded control system},
 description={An embedded system is a component made of some electronics and software that is installed into a piece of equipment to make it provide certain functionality. These components are often required to have a high reliability sometimes being required to operate for years at a time without any human intervention. In the modern world such items are found around us everywhere but often remain unnoticed until they go wrong.}
 }
 
\newacronym{ECS} % label 
{ECS} %abbreviation 
 {embedded control system} %long form 
 
\newglossaryentry{administrator}{
 name={administrator},
 description={The chief officer of the Federal Railroad Administration. That person has the authority to issue safety regulations and other emergency directives.}
 }
 
\newglossaryentry{runtime}{
 name={runtime},
 description={the time taken by an electric actuator to move from one defined position to another}
 }
 
\newglossaryentry{chief information officer}{
 name={chief information officer},
 description={}
 }
 
\newacronym{CIO} % label 
{CIO} %abbreviation 
 {chief information officer} %long form 
 
\newglossaryentry{software development}{
 name={software development},
 description={a set of activities that result in software products. Software development may include new development, modifications, reuse, reengineering, maintenance, or any other activities that result in software products.}
 }
 
\newglossaryentry{formal specification}{
 name={formal specification},
 description={a mathematical description of software or hardware that may be used to develop an implementation.}
 }
 
\newacronym{FS} % label 
{FS} %abbreviation 
 {formal specification} %long form 
 
\newglossaryentry{quality assurance}{
 name={quality assurance},
 description={part of quality management (3.2.8) focused on providing confidence that quality requirements (3.1.2) will be fulfilled}
 }
 
\newacronym{QA} % label 
{QA} %abbreviation 
 {quality assurance} %long form 
 
\newglossaryentry{source code}{
 name={source code},
 description={In computer science, source text (or source code) refers to the text of a computer programme written in a programming language that humans can read.}
 }
 
\newacronym{SC_3} % label 
{SC} %abbreviation 
 {source code} %long form 
 
\newglossaryentry{leverage}{
 name={leverage},
 description={}
 }
 
\newglossaryentry{model-checking}{
 name={model-checking},
 description={model checking explores all possible behaviours of a formal model to determine whether a user specified property is satisfied. In general a model check can have three outcomes. If the model checker determines that all the specified properties are satisfied, it will simply give an affirmative answer. In cases where the property is not satisfied, a counter-example illustrating where and how the property fails to hold is generated automatically. In some cases, a model checker may not be able to determine whether the given property is satisfied or not.}
 }
 
\newglossaryentry{consistency}{
 name={consistency},
 description={for a given measuring relay, the maximum value to be excepted within a given confidence level, of the difference between any two measured values determined under identical specified conditions}
 }
 
\newglossaryentry{expert}{
 name={expert},
 description={person who can judge the work assigned and recognise possible hazards on the basis of professional training, knowledge, experience and knowledge of the relevant equipment}
 }
 
\newglossaryentry{supplier}{
 name={supplier},
 description={organization (3.3.1) or person that provides a product (3.4.2)}
 }
 
\newglossaryentry{license}{
 name={license},
 description={An authorization issued by a Member State to an undertaking, by which its capacity as a railway undertaking is recognized. That capacity may be limited to the provision of specific types of services.}
 }
 
\newglossaryentry{sequence}{
 name={sequence},
 description={set of elementary instructions in a procedure that is to be totally executed in order to reach a functional objective}
 }
 
\newglossaryentry{communication}{
 name={communication},
 description={exchange or transfer of information}
 }
 
\newglossaryentry{semi-formal}{
 name={semi-formal},
 description={}
 }
 
\newacronym{SF} % label 
{SF} %abbreviation 
 {semi-formal} %long form 
 
\newglossaryentry{reuse}{
 name={reuse},
 description={Shall mean any operation by which packaging, which has been conceived and designed to accomplish within its life cycle a minimum number of trips or rotations, is refilled or used for the same purpose for which it was conceived, with or without the support of auxiliary products present on the market enabling the packaging to be refilled; such reused packaging will become packaging waste when no longer subject to reuse.}
 }
 
\newglossaryentry{improvement, optimisation}{
 name={improvement, optimisation},
 description={combination of all technical, administrative and managerial actions, intended to amelioate the reliability and/ or the maintainability and/ or the safety of an item, without changing the original function}
 }
 
\newglossaryentry{quality}{
 name={quality},
 description={a user perception of the attributes of a product}
 }
 
\newglossaryentry{observable validation}{
 name={observable validation},
 description={}
 }
 
\newacronym{OV} % label 
{OV} %abbreviation 
 {observable validation} %long form 
 
\newglossaryentry{programme}{
 name={programme},
 description={A documented set of time scheduled activities, resources and events serving to implement the organisational structure, responsibilities, procedures, activities,capabilities and resources that together ensure that an item will satisfy given RAM requirements relevant to a given contract or project.}
 }
 
\newglossaryentry{safety regulatory authority}{
 name={safety regulatory authority},
 description={Often a national government body responsible for setting or agreeing the safety requirements for a railway and ensuring that the railway complies with the requirements.}
 }
 
\newacronym{SRA} % label 
{SRA} %abbreviation 
 {safety regulatory authority} %long form 
 
\newglossaryentry{open source software}{
 name={open source software},
 description={Open source is a range of licences for software whose source code is publically available. Further developments are encouraged by the licence.}
 }
 
\newacronym{OSS} % label 
{OSS} %abbreviation 
 {open source software} %long form 
 
\newglossaryentry{railway support industry}{
 name={railway support industry},
 description={Generic term denoting supplier(s) of complete railwaysystems, their sub-systems or component parts.}
 }
 
\newacronym{RSI} % label 
{RSI} %abbreviation 
 {railway support industry} %long form 
 
\newglossaryentry{software baseline}{
 name={software baseline},
 description={released version of one or more work products, hardware or software items or elements configurations which is under configuration management and used as a basis for further development through change management process.}
 }
 
\newglossaryentry{safety integrity}{
 name={safety integrity},
 description={the ability of a safety-related system to achieve its required safety functions under all the stated conditions within a stated operational environment and within a stated period of time}
 }
 
\newacronym{SI} % label 
{SI} %abbreviation 
 {safety integrity} %long form 
 
\newglossaryentry{safety process}{
 name={safety process},
 description={the series of procedures that are followed to enable all safety requirements of a product to be identified and met}
 }
 
\newacronym{SP_2} % label 
{SP} %abbreviation 
 {safety process} %long form 
 
\newglossaryentry{signalling system}{
 name={signalling system},
 description={particular kind of system used on a railway to control and protect the operation of trains}
 }
 
\newglossaryentry{infrastructure}{
 name={infrastructure},
 description={
entirety/basis of personal, material and institutional establishments for the working of a specialised economy
}
 }
 
\newglossaryentry{failure}{
 name={failure},
 description={a deviation from the specified performance of a system. A failure is the consequence of a fault or error in the system}
 }
 
\newglossaryentry{fault}{
 name={fault},
 description={an abnormal condition that could lead to an error in a system. A fault can be random or systematic}
 }
 
\newglossaryentry{customer}{
 name={customer},
 description={organization (3.3.1) or person that receives a product}
 }
 
\newglossaryentry{simulator}{
 name={simulator},
 description={�
A debugging tool that runs on the host and pretends to be the target processor. A simulator can be used to test pieces of the embedded software before the embedded hardware is available. Unfortunately, attempts to simulate interactions with complex peripherals are often more trouble than they are worth.}
 }
 
\newglossaryentry{European Railway Agency}{
 name={European Railway Agency},
 description={The European Railway Agency, the European Community agency for railway safety and interoperability.}
 }
 
\newacronym{ERA} % label 
{ERA} %abbreviation 
 {European Railway Agency} %long form 
 
\newglossaryentry{safety case}{
 name={safety case},
 description={the documented demonstration that the product complies with the specified safety requirements}
 }
 
\newacronym{SC_2} % label 
{SC} %abbreviation 
 {safety case} %long form 
 
\newglossaryentry{domain specific modelling language}{
 name={domain specific modelling language},
 description={programming language or executable specification languages that offers expressiv power focused on a particular problem domain through dedicated notations and abstractions.}
 }
 
\newacronym{DSML} % label 
{DSML} %abbreviation 
 {domain specific modelling language} %long form 
 
\newglossaryentry{diesel multiple unit}{
 name={diesel multiple unit},
 description={A set of diesel-powered self-propelling passenger rail vehicles able to operate in multiple with other such sets. Such units, especially those consisting of a single vehicle, are sometimes termed railcars.}
 }
 
\newacronym{DMU} % label 
{DMU} %abbreviation 
 {diesel multiple unit} %long form 
 
\newglossaryentry{electric multiple unit}{
 name={electric multiple unit},
 description={A set of electrically powered self-propelling passenger rail vehicles able to operate in multiple with other such sets.}
 }
 
\newacronym{EMU} % label 
{EMU} %abbreviation 
 {electric multiple unit} %long form 
 
\newglossaryentry{real-time}{
 name={real-time},
 description={The ability to exchange or process information on specified events (such as arrival at a station, passing a station or departure from a station) on the trains journey as they occur.}
 }
 
\newglossaryentry{subset}{
 name={subset},
 description={set the elements of which all belong to a given set}
 }
 
\newglossaryentry{principle}{
 name={principle},
 description={essential rule, goal or attitude}
 }
 
\newglossaryentry{railway control}{
 name={railway control},
 description={}
 }
 
\newacronym{RC} % label 
{RC} %abbreviation 
 {railway control} %long form 
 
\newglossaryentry{modelling tool}{
 name={modelling tool},
 description={}
 }
 
\newacronym{MT} % label 
{MT} %abbreviation 
 {modelling tool} %long form 
 
\newglossaryentry{management system}{
 name={management system},
 description={A system for managing rail traffic, enabling it to operate on compatible signalling systems across European borders.}
 }
 
\newacronym{MS_2} % label 
{MS} %abbreviation 
 {management system} %long form 
 
\newglossaryentry{ecosystem}{
 name={ecosystem},
 description={the complex of living organisms, their physical environment, and all their interrelationships in a particular unit of space.}
 }
 
\newglossaryentry{rail vehicle}{
 name={rail vehicle},
 description={A vehicle suitable for circulation on its own wheels on railway lines, with or without traction.}
 }
 
\newacronym{RV} % label 
{RV} %abbreviation 
 {rail vehicle} %long form 
 
\newglossaryentry{management structure}{
 name={management structure},
 description={}
 }
 
\newacronym{MS} % label 
{MS} %abbreviation 
 {management structure} %long form 
 
\newglossaryentry{automatic train protection system}{
 name={automatic train protection system},
 description={A safety system that enforces either compliance with or observation of speed restrictions and signal aspects by trains.}
 }
 
\newacronym{ATPS} % label 
{ATPS} %abbreviation 
 {automatic train protection system} %long form 
 
\newglossaryentry{automatic code generation}{
 name={automatic code generation},
 description={function of automated tools allowing transformation of the application-oriented language into a form suitable for compilation or execution}
 }
 
\newacronym{ACG} % label 
{ACG} %abbreviation 
 {automatic code generation} %long form 
 
\newglossaryentry{hardware}{
 name={hardware},
 description={All parts of the computer that one could touch, including the keyboard, mouse, printer, external data carriers, graphic cards, etc.}
 }
 
\newacronym{HW} % label 
{HW} %abbreviation 
 {hardware} %long form 
 
\newglossaryentry{specification}{
 name={specification},
 description={document stating requirements}
 }
 
\newglossaryentry{communication-based train protection}{
 name={communication-based train protection},
 description={control-command system in North America similar to ERTMS/ETCS.}
 }
 
\newacronym{CBTP} % label 
{CBTP} %abbreviation 
 {communication-based train protection} %long form 
 
\newglossaryentry{software deployment}{
 name={software deployment},
 description={transferring, installing and activating a deliverable software baseline that has already been released and assessed}
 }
 
\newglossaryentry{advancing}{
 name={advancing},
 description={device attached to the support unit for moving the support forwards}
 }
 
\newglossaryentry{conformity}{
 name={conformity},
 description={fulfilment of a requirement}
 }
 
\newglossaryentry{infrastructure manager}{
 name={infrastructure manager},
 description={Any body or undertaking that is responsible in particular for establishing and maintaining railway infrastructure. This may also include the management of infrastructure control and safety systems. The functions of the infrastructure manager on a corridor or part of a corridor may be allocated to different bodies or undertakings}
 }
 
\newacronym{IM} % label 
{IM} %abbreviation 
 {infrastructure manager} %long form 
 
\newglossaryentry{standard}{
 name={standard},
 description={A standard approved by the European Committee for Standardization (CEN) or by the European Committee for Electrotechnical Standardization(Cenelec) as a ?European Standard (EN)? or ?Harmonization Document (HD)?, according to the common rules of those organizations, or by the European Telecommunications Standards Institute (ETSI) according to its own rules as a ?European Telecommunications Standard (ETS)?.}
 }
 
\newglossaryentry{monitoring}{
 name={monitoring},
 description={The systematic observation and recording of the performance of the train service and the infrastructure for the purpose of bringing about improvements in the performance of both.}
 }
 
\newglossaryentry{intelligent transportation system}{
 name={intelligent transportation system},
 description={a system that applies Information and Communication Technologies in the field of transport infrastructures and vehicles in order to improve different aspects such safety, traffic management, transportation times etc.}
 }
 
\newacronym{ITS} % label 
{ITS} %abbreviation 
 {intelligent transportation system} %long form 
 
\newglossaryentry{model-based development}{
 name={model-based development},
 description={fomalised application of modelling to support system requirements, design, analysis, verification and validation activities beginning in the conceptual design phase and continuing throughout development and later life cycle phases.}
 }
 
\newglossaryentry{certification}{
 name={certification},
 description={the legal recognition that a product, service, organisation or person complies with the applicable requirements.}
 }
 
\newglossaryentry{requirement}{
 name={requirement},
 description={need or exceptation that is stated, generally implied or obligatory}
 }
 
\newglossaryentry{software requirement}{
 name={software requirement},
 description={a description of what is to be produced by the software given the inputs and constraints. Software requirements include both high-level requirements and low-level requirements.}
 }
 
\newacronym{SR_2} % label 
{SR} %abbreviation 
 {software requirement} %long form 
 
\newglossaryentry{application}{
 name={application},
 description={A computer programme that performs a given task. Word processing and internet browsers are examples of applications.}
 }
 
\newglossaryentry{domain}{
 name={domain},
 description={}
 }
 
\newglossaryentry{software bug}{
 name={software bug},
 description={condition of a software item that may prevent it from performing as required}
 }
 
\newacronym{SB} % label 
{SB} %abbreviation 
 {software bug} %long form 
 
\newglossaryentry{error}{
 name={error},
 description={a deviation from the intended design which could result in unintended system behaviour or failure}
 }
 
\newglossaryentry{fault management}{
 name={fault management},
 description={}
 }
 
\newacronym{FM_3} % label 
{FM} %abbreviation 
 {fault management} %long form 
 
\newglossaryentry{bug management}{
 name={bug management},
 description={}
 }
 
\newacronym{BM_2} % label 
{BM} %abbreviation 
 {bug management} %long form 
 
\newglossaryentry{safety requirement}{
 name={safety requirement},
 description={stipulation referring exclusively to safety issues}
 }
 
\newacronym{SR} % label 
{SR} %abbreviation 
 {safety requirement} %long form 
 
\newglossaryentry{on-board unit}{
 name={on-board unit},
 description={on-board equipment for ETCS and the ETCS-related GSM-R.}
 }
 
\newacronym{OBU} % label 
{OBU} %abbreviation 
 {on-board unit} %long form 
 
\newglossaryentry{mistake}{
 name={mistake},
 description={human action or inaction that produces an unintended result}
 }
 
\newglossaryentry{Language for ETCS}{
 name={Language for ETCS},
 description={Harmonised rules within which messages can be transmitted and understood.}
 }
 
\newglossaryentry{utilisation}{
 name={utilisation},
 description={the name given to emanation in the construction environment}
 }
 
\newglossaryentry{hazard}{
 name={hazard},
 description={a condition that could lead to an accident}
 }
 
\newglossaryentry{damage}{
 name={damage},
 description={}
 }
 
\newglossaryentry{company}{
 name={company},
 description={recipient of a maintenance support service provided by the maintenance support service provider}
 }
 
\newglossaryentry{security}{
 name={security},
 description={}
 }
 
\newglossaryentry{danger}{
 name={danger},
 description={}
 }
 
\newglossaryentry{benchmark model}{
 name={benchmark model},
 description={}
 }
 
\newacronym{BM} % label 
{BM} %abbreviation 
 {benchmark model} %long form 
 
\newglossaryentry{Form Fit Function Interface Specification}{
 name={Form Fit Function Interface Specification},
 description={}
 }
 
\newacronym{FFFIS} % label 
{FFFIS} %abbreviation 
 {Form Fit Function Interface Specification} %long form 
 
\newglossaryentry{commercial off-the-shelf}{
 name={commercial off-the-shelf},
 description={Commercially available applications sold by vendors through public catalogue}
 }
 
\newacronym{COTS} % label 
{COTS} %abbreviation 
 {commercial off-the-shelf} %long form 
 
\newglossaryentry{operating system}{
 name={operating system},
 description={�
A piece of software that makes multitasking possible. An operating system typically consists of a set of system calls and a periodic clock tick ISR. The operating system is responsible for deciding which task should be using the processor at any given time and for controlling access to shared resources. See also real-time operating system, multitasking.}
 }
 
\newacronym{OS} % label 
{OS} %abbreviation 
 {operating system} %long form 
 
\newglossaryentry{assessor}{
 name={assessor},
 description={person or agent appointed to carry out the assessment}
 }
 
\newacronym{ASR} % label 
{ASR} %abbreviation 
 {assessor} %long form 
 
\newglossaryentry{designer}{
 name={designer},
 description={}
 }
 
\newacronym{DES} % label 
{DES} %abbreviation 
 {designer} %long form 
 
\newglossaryentry{European Union Public Licence}{
 name={European Union Public Licence},
 description={}
 }
 
\newacronym{EUPL} % label 
{EUPL} %abbreviation 
 {European Union Public Licence} %long form 
 
\newglossaryentry{Highly Recommended}{
 name={Highly Recommended},
 description={}
 }
 
\newacronym{HR} % label 
{HR} %abbreviation 
 {Highly Recommended} %long form 
 
\newglossaryentry{integrator}{
 name={integrator},
 description={}
 }
 
\newacronym{INT} % label 
{INT} %abbreviation 
 {integrator} %long form 
 
\newglossaryentry{Research and Development}{
 name={Research and Development},
 description={}
 }
 
\newacronym{RandD} % label 
{R \& D} %abbreviation 
 {Research and Development} %long form 
 
\newglossaryentry{Project Management Plan}{
 name={Project Management Plan},
 description={}
 }
 
\newacronym{PMP} % label 
{PMP} %abbreviation 
 {Project Management Plan} %long form 
 
\newglossaryentry{Requirements Manager}{
 name={Requirements Manager},
 description={}
 }
 
\newacronym{REQ} % label 
{REQ} %abbreviation 
 {Requirements Manager} %long form 
 
\newglossaryentry{not applicable}{
 name={not applicable},
 description={}
 }
 
\newacronym{NA} % label 
{NA} %abbreviation 
 {not applicable} %long form 
 
\newglossaryentry{Technical Specification for Interoperability}{
 name={Technical Specification for Interoperability},
 description={}
 }
 
\newacronym{TSI} % label 
{TSI} %abbreviation 
 {Technical Specification for Interoperability} %long form 
 
\newglossaryentry{System Configuration Management Plan}{
 name={System Configuration Management Plan},
 description={}
 }
 
\newacronym{SCMP} % label 
{SCMP} %abbreviation 
 {System Configuration Management Plan} %long form 
 
\newglossaryentry{tester}{
 name={tester},
 description={}
 }
 
\newacronym{TST} % label 
{TST} %abbreviation 
 {tester} %long form 
 
\newglossaryentry{Verification and Validation}{
 name={Verification and Validation},
 description={}
 }
 
\newacronym{VandV} % label 
{V \& V} %abbreviation 
 {Verification and Validation} %long form 
 
\newglossaryentry{intellectual property}{
 name={intellectual property},
 description={}
 }
 
\newacronym{IP} % label 
{IP} %abbreviation 
 {intellectual property} %long form 
 
\newglossaryentry{Work Package}{
 name={Work Package},
 description={}
 }
 
\newacronym{WP} % label 
{WP} %abbreviation 
 {Work Package} %long form 
 
\newglossaryentry{IP Clean}{
 name={IP Clean},
 description={}
 }
 
\newglossaryentry{validator}{
 name={validator},
 description={person or agent appointed to carry out validation}
 }
 
\newacronym{VAL} % label 
{VAL} %abbreviation 
 {validator} %long form 
 
\newglossaryentry{European Train Control System}{
 name={European Train Control System},
 description={A subset of ERTMS providing a level of protection against over speed and overrun depending upon the capability of the line side infrastructure.}
 }
 
\newacronym{ETCS} % label 
{ETCS} %abbreviation 
 {European Train Control System} %long form 
 
\newglossaryentry{verifier}{
 name={verifier},
 description={person or agent appointed to carry out verification}
 }
 
\newacronym{VER} % label 
{VER} %abbreviation 
 {verifier} %long form 
 
\newglossaryentry{European rail traffic management system}{
 name={European rail traffic management system},
 description={Signalling and operation management system using ETCS for the command control and GSM-R for data transmission.}
 }
 
\newacronym{ERTMS} % label 
{ERTMS} %abbreviation 
 {European rail traffic management system} %long form 
 
\newglossaryentry{Multifunction Vehicle Bus}{
 name={Multifunction Vehicle Bus},
 description={It is a part of the Train Communication Network (TCN), and it takes part in digital operation in the train. MVB is the bus part in each coach, and the Wire Train Bus (WTB) allows connecting the MVB parts with the train control system.}
 }
 
\newacronym{MVB} % label 
{MVB} %abbreviation 
 {Multifunction Vehicle Bus} %long form 
 
\newglossaryentry{safety integrity level}{
 name={safety integrity level},
 description={One of a number of defined discrete levels for specifying the safety integrity requirements of the safety functions to be allocated to the safety related systems. Safety Integrity Level with the highest figure has the highest level of safety integrity}
 }
 
\newacronym{SIL} % label 
{SIL} %abbreviation 
 {safety integrity level} %long form 
 
\newglossaryentry{strictly formal model}{
 name={strictly formal model},
 description={strictly model of Subset 26 using a semi-formal means of description}
 }
 
\newacronym{FFM} % label 
{FFM} %abbreviation 
 {strictly formal model} %long form 
 
\newglossaryentry{semi-formal model}{
 name={semi-formal model},
 description={model of Subset 26 using a semi-formal means of description}
 }
 
\newacronym{SFM} % label 
{SFM} %abbreviation 
 {semi-formal model} %long form 
 
\newglossaryentry{artifact}{
 name={artifact},
 description={The result of any activity in the software life-cycle such as requirements, architecture model, design specifications, source code and test scripts. A piece of information that is used or produced by a software development process. An artifact can be a model, a description, or software.}
 }
 
\newglossaryentry{traceability}{
 name={traceability},
 description={degree to which a relationship can be established between two or more products of a development process, especially those having a predecessor/successor or master/subordinate relationship to one another}
 }
 
\newglossaryentry{test procedure}{
 name={test procedure},
 description={A test procedure is a formal specification of test cases to be applied to one or more target program modules. Test procedures are executable.}
 }
 
\newacronym{TP_2} % label 
{TP} %abbreviation 
 {test procedure} %long form 
 
\newglossaryentry{formal methods}{
 name={formal methods},
 description={Formal methods are system design techniques that use rigorously specified mathematical models to build software and hardware systems.}
 }
 
\newacronym{FM_2} % label 
{FM} %abbreviation 
 {formal methods} %long form 
 
\newglossaryentry{functional architecture}{
 name={functional architecture},
 description={A functional architecture is an architectural model from a usage perspective}
 }
 
\newacronym{FA} % label 
{FA} %abbreviation 
 {functional architecture} %long form 
 
\newglossaryentry{formal model}{
 name={formal model},
 description={superordinate for semi-formal and strictly formal model}
 }
 
\newacronym{FM} % label 
{FM} %abbreviation 
 {formal model} %long form 
 
\newglossaryentry{preliminary hazard analysis}{
 name={preliminary hazard analysis},
 description={Preliminary hazard analysis (PHA) is a semi-quantitative analysis that is performed to: 1. Identify all potential hazards and accidental events that may lead to an accident; 2. Rank the identified accidental events according to their severity; 3. Identify required hazard controls and follow-up actions}
 }
 
\newacronym{PHA} % label 
{PHA} %abbreviation 
 {preliminary hazard analysis} %long form 
 
\newglossaryentry{software architecture}{
 name={software architecture},
 description={A set of artifacts (that is: principles, guidelines, policies, models, standards and processes) and the relationship between these artifacts that guide the selection, creation and implementation of solutions aligned with business goals.}
 }
 
\newacronym{SA} % label 
{SA} %abbreviation 
 {software architecture} %long form 
 
\newglossaryentry{code generation}{
 name={code generation},
 description={In computer science, code generation is the process by which a compiler's code generator converts some intermediate representation of source code into a form (e.g., machine code) that can be readily executed by a machine (often a computer).}
 }
 
\newacronym{CD} % label 
{CD} %abbreviation 
 {code generation} %long form 
 
\newglossaryentry{tool chain}{
 name={tool chain},
 description={A set of software utilities used to perform an operation. For example, in program development, the toolchain to turn source code into a working machine language program includes a compiler, assembler, linker and debugger.}
 }
 
\newacronym{TC_2} % label 
{TC} %abbreviation 
 {tool chain} %long form 
 
\newglossaryentry{deliverable product}{
 name={deliverable product},
 description={a thing able to be provided, especially as a product of a development process}
 }
 
\newacronym{DP} % label 
{DP} %abbreviation 
 {deliverable product} %long form 
 
\newglossaryentry{system requirement specification}{
 name={system requirement specification},
 description={}
 }
 
\newacronym{SRS} % label 
{SRS} %abbreviation 
 {system requirement specification} %long form 
 
\newglossaryentry{test case}{
 name={test case},
 description={the test case is a sequence which checks if the system satisfies a requirement}
 }
 
\newacronym{TC} % label 
{TC} %abbreviation 
 {test case} %long form 
 
\newglossaryentry{functional interface specification}{
 name={functional interface specification},
 description={Specification describing the properties of an interface between components of a piece of equipment that ensures operational interoperability}
 }
 
\newacronym{FIS} % label 
{FIS} %abbreviation 
 {functional interface specification} %long form 
 
\newglossaryentry{executable code}{
 name={executable code},
 description={Software in a form that can be run in the computer and typically refers to machine language, which is the set of native instructions the computer carries out in hardware.}
 }
 
\newacronym{EC} % label 
{EC} %abbreviation 
 {executable code} %long form 
 
\newglossaryentry{tolerable hazard rate}{
 name={tolerable hazard rate},
 description={The tolerable hazard rate is the target measure with respect to both systematic and random failure integrity and therefore the result of the risk analysis.}
 }
 
\newacronym{THR} % label 
{THR} %abbreviation 
 {tolerable hazard rate} %long form 
 
\newglossaryentry{verification}{
 name={verification},
 description={Confirmation by examination and provision of objective evidence that the specified requirements have been fulfilled.}
 }
 
\newglossaryentry{software component}{
 name={software component},
 description={Program modules that are designed to interoperate with each other at runtime.}
 }
 
\newacronym{SC} % label 
{SC} %abbreviation 
 {software component} %long form 
 
\newglossaryentry{sub-system}{
 name={sub-system},
 description={a portion of a system which fulfils a specialised function}
 }
 
\newacronym{SS} % label 
{SS} %abbreviation 
 {sub-system} %long form 
 
\newglossaryentry{application programming interface}{
 name={application programming interface},
 description={an abstraction that is defined by the description of an interface and the behaviour of the interface.}
 }
 
\newacronym{API} % label 
{API} %abbreviation 
 {application programming interface} %long form 
 
\newglossaryentry{safety properties}{
 name={safety properties},
 description={A safety property is a system property which states that something will not happen.}
 }
 
\newacronym{SP} % label 
{SP} %abbreviation 
 {safety properties} %long form 
 
\newglossaryentry{European Vital Computer}{
 name={European Vital Computer},
 description={Computer device for the onboard ETCS.}
 }
 
\newacronym{EVC} % label 
{EVC} %abbreviation 
 {European Vital Computer} %long form 
 
\newglossaryentry{tool platform}{
 name={tool platform},
 description={Collaborative development environment of a tool chain}
 }
 
\newacronym{TP} % label 
{TP} %abbreviation 
 {tool platform} %long form 
 
\newglossaryentry{International Software Testing Qualifications Board}{
 name={International Software Testing Qualifications Board},
 description={The International Software Testing Qualifications Board is a software testing qualification certification organisation that operates internationally}
 }
 
\newacronym{ISTQB} % label 
{ISTQB} %abbreviation 
 {International Software Testing Qualifications Board} %long form 
 

 %End of Glossary Terms
 %Glossary entries: 179 
 %Abbreviations entries: 102 
